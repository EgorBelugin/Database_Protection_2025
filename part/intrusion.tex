\section{Распознавание вторжений в БД}
\subsection{Основные понятия}

\textbf{Обнаружение вторжений} -- это сбор и анализ информации из различных точек защищаемой компьютерной системы (вычислительной сети) для выявления как попыток нарушения, так и реальных нарушений защиты (вторжений).


Среди методов, используемых в подсистемах анализа современных систем обнаружения вторжений (СОВ), можно выделить два направления:
\begin{itemize}
	\item обнаружение аномалий в защищаемой системе
	\item поиск злоупотреблений
\end{itemize}

\textbf{Цели выявления злоупотреблений} -- поиск шаблонов известных атак. В качестве искомого шаблона может выступать последовательность событий, паттерны сетевого трафика различных протоколов, определенные ассемблерные команды и т.д.

\textbf{Место процедуры распознавания вторжений в общей системе защиты} -- обнаружение и регистрация атак, а также оповещение при срабатывании определенного правила.

\subsection{Системы распознавания вторжений}

\textbf{Системы обнаружения и предотвращения вторжений (IPS/IDS)} -- это комплекс программных или аппаратных средств, которые выявляют факты и предотвращают попытки несанкционированного доступа в корпоративную систему.

\subsubsection*{Типы моделей систем распознавания вторжений (ID-систем)}

Системы обнаружения и предотвращения разделяют на:

\begin{itemize}
	\item \textbf{Сетевые (NIDS)} -- проверке подвергается сетевой трафик с концентратора или коммутатора
	\item \textbf{Основанные на протоколах (PIDS)} -- предполагает наблюдение за HTTP- и HTTPS-протоколами
	\item \textbf{Основанные на прикладных протоколах (APIDS)} -- в таких системах проверяются специализированные прикладные протоколы. Например, на веб-сервере с SQL базой данных СОВ будет отслеживать содержимое SQL команд, передаваемых на сервер.
	\item \textbf{Узловые Host-Based (HIDS)} -- подвергают анализу журналы приложений, состояния хостов, а также системные вызовы.
	\item \textbf{Гибридные} -- являются композицией нескольких видов систем обнаружения вторжений.
\end{itemize}

\paragraph{Общая структура ID-систем} 

\subsubsection*{Концептуальная схема систем обнаружения вторжений включает в себя:}
\begin{itemize}
	\item Подсистему сбора событий (сенсорную)
	\item Подсистему анализа данных, полученных от сенсорной подсистемы
	\item Подсистему хранения событий
	\item Консоль администрирования
\end{itemize}

\paragraph{Шаблоны классов пользователей}

\paragraph{Модели известных атак}
\begin{itemize}
	\item Необычные запросы и команды администрирования SQL сервера, часто употребляемые взломщиками (DROP, CREATE и т.д.)
	\item Незаконные операции DELETE, UPDATE и INSERT
	\item Запросы, содержащие id администратора
	\item Обращение к служебным таблицам и скрытым данным
	\item Запросы, всегда возвращающие TRUE
	\item Попытка обнулить поля с паролем
	\item Внедрение в запросе OR, сравнения констант и другое
\end{itemize}

\subsection{Экспертные ID-системы}

Технологию построения экспертных систем часто называют инженерией знаний. Как правило, этот процесс требует специфической формы взаимодействия создателя экспертной системы, которого называют инженером знаний, и одного или нескольких экспертов в некоторой предметной области.  Инженер знаний <<извлекает>> из экспертов процедуры, стратегии,  эмпирические правила, которые они используют при решении задач, и  встраивает эти знания в экспертную систему. В результате появляется   компьютерная программа, которая решает задачи во многом так же, как   эксперты -- люди \autocite{ExpertSystems}.

В экспертных системах могут использоваться импликационные правила (Если [условие] то [действие]). Например:
\begin{itemize}
	\item ЕСЛИ с одного узла за время T поступает N пакетов, ТО записать в лог факт: происходит DoS атака (факт А)
	\item Если наблюдается более чем N фактов А, ТО записать в лог факт: происходит DDoS атака
\end{itemize}

\paragraph{Метрики}

\begin{itemize}
	\item \textbf{Показатель активности} -- величина, при превышении которой активность подсистемы оценивается как быстро прогрессирующая. В общем случае используется для обнаружения аномалий, связанных с резким ускорением в работе. Пример: среднее число записей аудита, обрабатываемых для элемента защищаемой системы в единицу времени.
	\item \textbf{Распределение активности в записях аудита} -- распределение во всех типах активности в актуальных записях аудита. Здесь под активностью понимается любое действие в системе, например, доступ к файлам, операции ввода-вывода.
	\item \textbf{Измерение категорий} -- распределение определенной активности в категории\footnotemark. Например, относительная частота количества регистраций в системе (логинов) из каждого физического места нахождения. Предпочтения в использовании программного обеспечения системы (почтовые службы, компиляторы, командные интерпретаторы, редакторы и т.д.)
	\item \textbf{Порядковые измерения} -- используется для оценки активности, поступающей в виде цифровых значений. Например, количество операция ввода-вывода, инициируемых каждым пользователем. Порядковые изменения вычисляют общую числовую статистику значений определенной активности, в то время как измерение категорий подсчитывает количество активностей.
\end{itemize}

\footnotetext{Здесь под \textit{категорией} понимается группа подсистем, объединенных по некоему общему принципу}

\subsubsection*{Статистические модели}

При обнаружении аномалий с использованием профайла в основном применяют статистические методы оценки. Процесс обнаружения происходит следующим образом: текущие значения измерений профайла сравнивают с сохраненными значениями. Результат сравнения - показатель аномальности в измерении. Общий показатель аномальности в простейшем случае может вычисляться при помощи некоторой общей функции от значений показателя аномалии в каждом измерении профайла.

Например, пусть $M_1, M_2, \dots, M_n$ -- измерения профайла, а $S_1, S_2, \dots, S_n$ -- соответственно представляют собой значения аномалии каждого из измерений. Чем больше число $S_i$, тем больше аномалии в $i$-ом показателе. Объединяющая функция может быть взвешенной суммой их квадратов:

\begin{equation}
	a_1s_1^2 + a_2s_2^2 + \dots + a_ns_n^2 > 0,
\end{equation}

где $a_i$ -- отражает относительный вес метрики $M_i$.

Параметры $M_1, M_2, \dots, M_n$ могут быть зависимыми друг от друга. В таком случае, объединяющая функция будет более сложной.

\subsubsection*{Профили}

Одним из способов формирования <<образа>> нормального поведения системы состоит в накоплении измерений значения параметров оценки в специальной структуре данных. Эта структура данных называется \textit{профайлом}. 

Основные требования, предъявляемые к структуре профайла:
\begin{itemize}
	\item Минимальный конечный размер
	\item Быстрое выполнение операции обновления
\end{itemize}

\paragraph{Примеры ID-систем}
\begin{enumerate}
	\item \textbf{GreenSQL-FW}. Работает как прокси-сервер между веб-приложением и SQL сервером. Анализирует SQL команды на предмет аномальных запросов.
	
	GreenSQL поддерживает несколько режимов работы:
	\begin{itemize}
		\item \textbf{Simulation Mode} -- пассивная система обнаружения атак (IDS). Протоколирует SQL запросы, выдает предупреждения на консоль управления.
		\item \textbf{Blocking Suspicious Commands} -- активная СОА. Атаки не только обнаруживаются, но и блокируются (IPS) в соответствии с установленными правилами, указывающими на аномальность запроса.
		\item \textbf{Active protection from unknown queries} -- блокирование всех неизвестных запросов (db firewall)
		\item \textbf{Learning mode} -- предназначен для прогона и настройки правил в <<чистой>> среде, что позволяет сформировать белый список и предотвратить в последствии ложные срабатывания анализатора запросов.
	\end{itemize}
	
	\item \textbf{Snort} -- свободная сетевая система предотвращения вторжений (IPS) и обнаружения вторжений (IDS) с открытым исходным кодом, способная выполнять регистрацию пакетов и в реальном времени осуществлять анализ трафика в IP-сетях. Способна выявлять атаки на SQL базы данных. 

	Особенности Snort:
	\begin{itemize}
		\item Возможность написания собственных правил
		\item Расширение функциональности с помощью подключения дополнительных модулей
		\item Гибкая система оповещения об атаках: Log-файлы, устройства вывода, БД и прочие
	\end{itemize}
	
\end{enumerate}

\subsection{Развитие систем распознавания вторжений}

Дальнейшие направления совершенствования связаны с внедрением в теорию и практику СОВ общей теории систем, методов синтеза и анализа информационных систем, конкретного аппарата теории распознавания образов. Эти разделы теории предполагают получение конкретных методов исследования для области систем СОВ.

До настоящего времени не описана СОВ как подсистема информационной системы в терминах общей теории систем. Необходимо обосновать показатель качества СОВ, элементный состав СОВ, ее структуру и взаимосвязи с информационной системой.

В связи с наличием значительного количества факторов различной природы, слаженная работа информационной системы и СОВ имеет вероятностный характер. Вследствие этого актуальным является обоснование вида вероятностных законов конкретных параметров функционирования. Особо следует выделить задачу обоснования функции потерь информационной системы, задаваемую в соответствии с ее целевой функцией на области параметров функционирования системы. При этом целевая функция должна быть определена не только на экспертном уровне, но и в соответствии с совокупностью параметров функционирования всей информационной системы и задачами, возложенными на нее. В таком случае показатель качества СОВ будет определяться как один из параметров, влияющих на целевую функцию, а его допустимые значения -- допустимыми значениями функции потерь.

После обоснования законов и функций, реальной задачей является получение оптимальной структуры СОВ в виде совокупности математических операций с помощью формализованных методов. Таким образом, может быть решена задача синтеза структуры СОВ. На основе полученных математических операций можно будет рассчитать зависимости показателей качества функционирования СОВ от параметров ее функционирования, а также от параметров функционирования информационной системы, то есть, будет возможен реальный анализ качества функционирования СОВ.

Сложность применения формализованного аппарата анализа и синтеза информационных систем к СОВ заключается в том, что конкретные реализации информационного комплекса и его подсистемы - СОВ состоят из разнородных элементов, которые могут описываться различными разделами теории (системами массового обслуживания, конечными автоматами, теорией вероятности, теорией распознавания образов и т.д.), то есть, рассматриваемый объект исследования является составным. В результате, математические модели, по-видимому, можно получить только для отдельных составных частей СОВ, что затрудняет анализ и синтез СОВ в целом. Однако, дальнейшая конкретизация применения формализованного аппарата анализа и синтеза позволит оптимизировать СОВ.

На основе изложенного можно сделать вывод о наличии в практической среде значительного опыта решения проблем обнаружения вторжений. Применяемые СОВ в значительной степени основаны на эмпирических схемах процесса обнаружения вторжений. Дальнейшее совершенствование СОВ связано с конкретизацией методов синтеза и анализа сложных систем, теории распознавания образов в применении к СОВ.