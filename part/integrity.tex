\section{Механизмы обеспечения целостности СУБД}
3.1. Угрозы целостности СУБД.
Основные виды и причины возникновения угроз целостности. Способы противодействия.

3.2. Метаданные и словарь данных.
Назначение словаря данных. Доступ к словарю данных. Состав словаря. Представления словаря.

3.3. Понятие транзакции.
Фиксация транзакции. Прокрутки вперед и назад. Контрольная точка. Откат. Транзакции как средство изолированности пользователей. Сериализация транзакций. Методы сериализации транзакций.

3.4. Блокировки.
Режимы блокировок. Правила согласования блокировок. Двухфазный протокол синхрони-зационных блокировок. Тупиковые ситуации, их распознавание и разрушение.

3.5. Ссылочная целостность.
Декларативная и процедурная ссылочные целостности. Внешний ключ. Способы поддер-жания ссылочной целостности. 

3.6. Правила(триггеры).
Цели использования правил. Способы задания, моменты выполнения.

3.7. События.
Назначение механизма событий. Сигнализаторы событий. Типы уведомлений о происхож-дении события. Компоненты механизма событий.