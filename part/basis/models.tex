\subsection{Модели безопасности в СУБД}

Сегодня существует большое число различных теоретических моделей, которые позволяют описать 
практически все аспекты безопасности и обеспечивают средства защиты информации формально 
подтверждённой алгоритмической базой. Однако на практике воспользоваться результатами данных 
исследований не всегда удается, потому что слишком часто теория защиты информации не согласуется 
с реальной жизнью. Дело в том, что теоретические исследования в области защиты информационных 
систем носит разрозненный характер и не составляет комплексной теории. Существующие технические 
разработки основаны на различных подходах проблеме, поэтому методы её решения существенно 
различаются. Наибольшее развитие получили два подхода -- это формальное моделирования политики 
безопасности и криптография. Эти различные по происхождению и решаемым задачам подходы взаимно 
дополняют друг друга. В отличии от криптографии, формальные модели безопасности предоставляют 
разработчикам защищенных систем принципы, которые лежат в основе архитектуры защищенной системы 
и определяют концепцию её построения \autocite{Zegzhda}.

Напомним, что \textit{под политикой безопасности понимаем совокупность норм и правил}, которые 
регламентируют процесс обработки информации. Выполнение этих правил обеспечивает защиту от 
определённого множества угроз и составляет необходимые условия безопасности систем. Формальным 
выражением политики безопасности будем называть \textbf{моделью политики безопасности} (или 
просто моделью безопасности).

Для чего необходимы модели безопасности? Обычно модель безопасности в себя включает:
\begin{itemize}
    \item определение условий, которым должно подчиняться поведение системы,
    \item выработка критериев безопасности.
\end{itemize}
\textbf{\textit{Основная цель создания формальный модели политики безопасности информационной 
системы}} -- это проведение формального доказательства соответствия системы критериям при 
соблюдении установленных правил и ограничений.

На практике это означает, что только соответствующим образом уполномоченные пользователи могут 
получить доступ к информации, смогут осуществлять с ней только санкционированные политикой 
безопасности действия.

\paragraph{Основные представления информационных систем}

Перед расcмотрением наиболее распространённых моделей политик безопасности, основанных на 
контроле доступа субъектов к объектам, введём основные представления систем \autocite{Zegzhda}:
\begin{enumerate}
    \item Сущности системы -- \textbf{субъекты и объекты}. Интуитивно объекты можно представить в 
    виде ячеек, содержащих информацию, а субъектами можно считать выполняющиеся программы, 
    которые воздействует на объекты различными способами. Безопасность обработки информации 
    обеспечивается с помощью решения задачи управления доступом субъектов к объектам в соответствии 
    с политикой безопасности

    \item Взаимодействия в системе моделируются с помощью установления \textbf{отношений определённых 
    типов между субъектами и объектами}. Множество типов отношений определяется в виде набора 
    операций, которые субъекты могут проводить над объектами

    \item Все операции контролируются \textbf{монитором взаимодействий}, разрешаются или запрещаются в 
    соответствии с политикой безопасности. Монитор безопасности –- механизм реализации политики 
    безопасности в автоматизированной системе, совокупность аппаратных, программных и специальных 
    компонент системы, реализующих функции защиты и обеспечения безопасности \autocite{URFULecture10Models}

    \item \textbf{Политика безопасности} задается в виде правил, в соответствии с которыми должны 
    осуществляться все взаимодействия между субъектами и объектами

    \item Совокупность множеств субъектов, объектов и отношений между ними определяет \textbf{текущее 
    состояние системы}

    \item Основной элемент модели безопасности -- \textbf{доказательство утверждения (теоремы)} о 
    том, что система, находящаяся в безопасном состоянии, не может перейти в небезопасное состояние 
    при соблюдении всех установленных правил и ограничений
\end{enumerate}

Отметим, что среди моделей политик безопасности можно выделить два основных направления (по принципу 
установления отношений между субъектами и объектами):
\begin{enumerate}
    \item дискреционные,
    \item мандатные политики.
\end{enumerate}

\subsubsection{Дискреционная (избирательная) модель безопасности}

Политика дискреционного (избирательного) доступа реализована в большинстве защищенных информационных 
систем и исторически является первой проработанной в теоретическом и практическом плане. Первые описания 
моделей дискреционного доступа появились еще в 60-х годах \autocite{URFULecture10Models}. Позднее 
будут рассмотрены наиболее популярные модели.

Важно отметить, что модели дискреционного доступа непосредственно основываются на субъектно-объектной 
модели информационной системы введенной ранее и развивают ее как совокупность некоторых множеств 
взаимодействующих элементов (субъектов, объектов и т. д.). Множество (область) безопасных доступов 
в моделях дискреционного доступа определяется дискретным набором троек «пользователь (субъект) –- поток
(операция) –- объект» \autocite{URFULecture10Models}.

\paragraph{Дискреционная модель на примере матричной}

На практике наибольшее применение получили дискреционные модели, основанные на матрице доступа. 
Дискреционная (матричная) модель является наиболее простой и распространённой. Отношения между 
субъектами и объектами можно представить в виде матрицы доступа (access matrix), в строках которой 
перечислены субъекты, в заголовках столбцов перечислены объекты, а в ячейках (на пересечении строк 
и столбцов) указываются разрешенные виды доступа. То есть задаются права доступа для каждого сочетания 
«субъект+объект». Виды доступа могут быть определены в каждом случае индивидуально. Обычно в матрице 
используются следующие обозначения: w –- «писать», r –- «читать», e –- «исполнять».

\paragraph{Недостатки матричной модели}

Недостатками матричной модели являются ее статичность и чрезмерно детализированный способ указания 
отношений между субъектами и объектами. При большом количестве пользователей администрирование такой 
системы усложняется, что приводит к возникновению ошибок.

Однако основной недостаток матричной модели заключается в том, что субъект, имеющий право на чтение 
информации может передать эту информацию другим субъектам, без уведомления владельца объекта. Причина 
этого недостатока исходит из того, что \textbf{\textit{во всех дискреционных моделях контролируются только 
операции доступа субъектов к объектам, а не потоки информации между ними}}. Поэтому, когда происходит 
перенос информации из доступного пользователю объекта в объект, доступный нарушителю, то формально 
никакое правило дискреционной политики безопасности не нарушается, но утечка информации происходит 
\autocite{URFULecture10Models}. Кроме того, не всегда можно назначить владельца каждому объекту 
(объекты часто принадлежат не отдельным субъектам, а всей системе).

Статичность модели препятствует быстрому внесению изменений в систему, например, при увольнении/найме 
новых сотрудников или при смене их должности/переводу в другое подразделение. Кроме того, для получения 
полного доступа к системе злоумышленнику достаточно узнать данные учетной записи пользователя. 

\paragraph{Централизованный и распределенный подходы построения матрицы}

Принцип организации матрицы доступа в реальных системах определяет использование двух подходов –- 
централизованного и распределенного.

При централизованном подходе матрица доступа создается как отдельный самостоятельный объект с особым 
порядком размещения и доступа к нему. Количество объектов и субъектов доступа в реальных АС может быть
велико. Для уменьшения количества столбцов матрицы объекты доступа АС могут делиться на две группы –- 
группу объектов, доступ к которым не ограничен, и группу объектов дискреционного доступа. В матрице 
доступа представляются права пользователей только к объектам второй группы. Наиболее известным примером 
такого подхода являются «биты доступа» в UNIX-системах \autocite{URFULecture10Models}.

При распределенном подходе матрица доступа как отдельный объект не создается, а представляется или 
«списками доступа», распределенными по объектам системы, или «списками возможностей», распределенными 
по субъектам доступа. В первом случае каждый объект системы, помимо идентифицирующих характеристик, 
наделяется еще своеобразным списком, непосредственно связанным с самим объектом и представляющим, 
по сути, соответствующий столбец матрицы доступа. Во втором случае список с перечнем разрешенных для 
доступа объектов (строку матрицы доступа) получает каждый субъект при своей инициализации 
\autocite{URFULecture10Models}.

\subsubsection{Мандатная (полномочная) модель безопасности}

Мандатные (многоуровневые) модели доступа были разработаны с целью устранения недостатков дискреционных 
(матричных) моделей. Политика мандатного доступа является примером использования технологий, 
наработанных во внекомпьютерной сфере, в частности принципов организации секретного делопроизводства 
и документооборота, применяемых в государственных структурах большинства стран.

Основным положением политики мандатного доступа является назначение: (1) всем участникам процесса 
обработки защищаемой информации и (2) документам, в которых она содержится, специальной метки 
(например \textit{секретно}, \textit{сов. секретно} и т. д.) получившей название \textbf{уровня безопасности}. 
Все уровни безопасности упорядочиваются с помощью установленного \textbf{отношения
доминирования}, например, уровень безопасности \textit{сов. секретно} считается более высоким чем
уровень \textit{секретно}.

В многоуровневой модели задается порядок следования уровней доступа и уровней безопасности (секретности), 
а затем устанавливаются \textbf{связи между уровнем доступа и уровнем секретности}. Эти связи, 
устанавливающие разрешение на доступ, называются \textbf{мандатными связями}.

Контроль доступа осуществляется в зависимости от уровней безопасности взаимодействующих сторон на 
основании двух правил \autocite{URFULecture10Models}:
\begin{enumerate}
    \item \textbf{No read up (NRU)} – нет чтения вверх: субъект имеет право читать только те документы, 
    уровень безопасности которых не превышает его собственный уровень безопасности

    \item \textbf{No write down (NWD)} – нет записи вниз: субъект имеет право заносить информацию только 
    в те документы, уровень безопасности которых не ниже его собственного уровня безопасности
\end{enumerate}
\textbf{\textit{Первое правило обеспечивает защиту информации, обрабатываемой более доверенными 
(высокоуровневыми) лицами, от доступа со стороны менее доверенных (низкоуровневых). Второе правило 
предотвращает утечку информации (сознательную или несознательную) от высокоуровневых участников 
процесса обработки информации к низкоуровневым.}}

Таким образом, при создании системы безопасности с использованием мандатной модели необходимо:
\begin{itemize}
    \item Определить все объекты в системе, к которым необходимо предоставить доступ;
    \item Составить список субъектов, получающих доступ к объектам;
    \item Разбить все объекты на группы по уровню конфиденциальности (секретности);
    \item Создать группы для субъектов, различающиеся по уровню доступа (установить формы допуска);
    \item Установить для каждого субъекта уровень доступа (выдать им формы допуска);
    \item Определить все возможные виды разрешений на доступ;
    \item Установить связи (в виде разрешений на доступ) между группами субъектов и группами объектов.
\end{itemize}

\textit{Преимуществом мандатной модели безопасности перед дискреционной состоит в возможности 
контролировать не только операции доступа субъектов к объектам, но и потоки информации между 
ними (за счёт наличия дополнительных правил накладываемых на мандатные связи)}.

\subsubsection{Классификация моделей безопасности}

Обычно говоря о моделях безопасности имеют ввиду модели разграничения доступа. Можно выделить три 
основные категории субъектно-объектных моделей разграничения доступа:
\begin{itemize}
    \item Модели систем дискреционного разграничения доступа;
    \item Модели систем мандатного разграничения доступа;
    \item Модели ролевого разграничения доступа.
    % \item Модели безопасности информационных потоков;
    % \item Субъектно-ориентированная модель изолированной программной среды.
\end{itemize}

Модели систем дискреционного разграничения доступа и модели систем мандатного разграничения доступа 
ранее мы уже рассмотрели, дадим краткое описание ролевым моделям безопасности.

\paragraph{Модели ролевого разграничения доступа}

Ролевые модели безопасности нельзя отнести ни к дискреционным, ни к мандатным моделям, потому что 
управление доступом в них осуществляется как на основе матрицы прав доступа для ролей, так и с помощью 
правил, регламентирующих назначение ролей пользователям и их активации во время сеансов. Поэтому 
ролевая модель представляет собой совершенно особый тип политики, основанный на компромиссе между 
гибкостью и простотой управления доступом (характерным для дискреционных моделей), так и жёсткостью 
правил контроля доступа (соответствующие мандатным моделям).

В ролевой модели классическое понятие \textit{субъект} заменяется понятиями \textit{пользователь} и 
\textit{роль}. Пользователь -- это человек, работающий c системой и выполняющий определённые обязанности. 
Роль -- это активно действующая в системе абстрактная сущность, c которой связан ограниченный, логический 
связный набор полномочий, необходимый для осуществления определённой деятельности.

\subsubsection{Аспекты исследования моделей безопасности}
\subsubsection{Особенности применения моделей безопасности в СУБД}

\subsubsection{Дискреционные модели}
\paragraph{HRU}
\paragraph{Take-Grant}
\paragraph{Action-Entity}
\paragraph{Wood}

\subsubsection{Мандатные модели}
\paragraph{Bell-LaPadula}
\paragraph{Biba}
\paragraph{Dion}
\paragraph{Sea View}
\paragraph{Jajodia\&Sandhu}
\paragraph{Smith\&Winslett}
\paragraph{Решеточная}

\subsubsection{БД с многоуровневой секретностью (MLS)}
\subsubsection{Многозначность}