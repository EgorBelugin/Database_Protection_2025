\subsection{Критерии защищенности БД}

\subsubsection{Критерии оценки надежных компьютерных систем (TCSEC)}
<<Критерии оценки безопасности компьютерных систем>> (Trusted Computer System Evaluation Criteria),
получившие неформальное название <<Оранжевая книга>> \cite{OrangeBook}, были разработаны Министерством обороны
США в 1983 году с целью определения требований безопасности, предъявляемых к аппаратному,
программному и специальному обеспечени	ю компьютерных систем и выработки соответствующей
методологии и технологии анализа степени поддержки политики безопасности в компьютерных
системах военного назначения. В данном документе были впервые нормативно определены такие
понятия, как <<политика безопасности>>, <<ядро безопасности>>, <<доверенная компьютерная база>>
\footnote [1]{ Альтернативный вариант перевода --- <<доверенная вычислительная система>> } (\textit{англ.} The Trusted Computing Base, TCB).

Предложенные в этом документе концепции защиты и набор функциональных требований послужили
основой для формирования всех появившихся впоследствии стандартов безопасности.

В <<Оранжевой книге>> предложены три категории требований безопасности --- политика безопасности,
подотчетность и корректность, в рамках которых сформулированы шесть базовых требований безопасности.
Первые четыре требования направлены непосредственно на обеспечение безопасности информации,
а два последних --- на качество самих средств защиты.

Рассмотрим эти требования подробнее \cite{OrangeBook}:

\begin{enumerate}
	\item \textbf{Политика безопасности}
	\begin{itemize}
		\item \textbf{Политика безопасности}\\
		В системе должна существовать явная и четко определённая политика безопасности. Возможность предоставления субъектам доступа к объектам должна определяться на основе
		их идентификации и набора правил управления доступом. Там, где необходимо, должна использоваться политика мандатного управления доступом, позволяющая эффективно реализовать
		разграничение доступа к информа­ции различного уровня конфиденциальности. Кроме того, необходимы дискреционные средства контроля безопасности для обеспечения доступа к данным для
		уполномоченных пользователей или групп пользователей.
		\item \textbf{Метки}\\
		С объектами должны быть ассоциированы метки безопасности (метки доступа), используемые в качестве атрибутов контроля доступа. Для реализации мандатного управления доступом система 
		должна обеспечивать возможность присваивать каждому объекту метку или набор атрибутов, определяющих  степень конфиденциальности (гриф секретности) объекта и/или режимы доступа к 
		этому объекту.
	\end{itemize}
	\item \textbf{Подотчетность}
	\begin{itemize}
		\item \textbf{Идентификация и аутентификация}\\
		Все субъекты должны иметь уникальные идентификаторы. Контроль доступа должен осуществляться на основании результатов идентификации субъекта и объекта доступа, подтверждения 
		подлинности их идентификаторов (аутентификации) и правил разграничения доступа. Данные, используемые для идентификации и аутентификации, должны быть защищены от несанкционированного 
		доступа, модификации и уничтожения и должны быть ассоциированы со всеми активными компонентами компьютерной системы, функционирование которых критично с точки зрения безопасности.
		\item \textbf{Регистрация и учет}\\
		Для определения степени ответственности пользователей за действия в системе, все происходящие в ней события, имеющие значение с точки зрения безопасности, должны отслеживаться и 
		регистрироваться в защищенном протоколе. Система регистрации должна осуществлять анализ общего потока событий и выделять из него только те события, которые оказывают влияние 
		на безопасность, для сокращения накладных расходов на дальнейший аудит и повышения эффективности анализа. Подледащие аудиту данные должны быть надежно защищены от несанкционированного 
		доступа, модификации и уничтожения.
	\end{itemize}
	\item \textbf{Корректность}
	\begin{itemize}
		\item \textbf{Контроль корректности}\\
		Средства защиты должны содержать независимые аппаратные и/или программные компоненты, обеспечивающие работоспособность функций защиты. Это означает, что все средства защиты, 
		обеспечивающие политику безопасности, управление атрибутами и метками безопасности, идентификацию и аутентификацию, регистрацию и учёт, должны находиться под контролем средств, 
		проверяющих корректность их функционирования. Основной принцип контроля корректности состоит в том, что средства контроля должны быть полностью независимы от средств защиты.
		\item \textbf{Непрерывность защиты}\\
		Все средства защиты (в т.ч. и реализующие данное требование) должны быть защищены от несанкционированного вмешательства и/или отключения, причем эта защита должна быть постоянной 
		и непрерывной в любом режиме функционирования системы защиты и компьютерной системы в целом. Данное требование распространяется на весь жизненный цикл компьютерной системы. 
		Кроме того, его выполнение является одним из ключевых аспектов формального доказательства безопасности системы.
	\end{itemize}
\end{enumerate}

В рамках этих базовых требований по безопасности в <<Оранжевой книге>> сформулированы критерии оценки безопасности компьютерной системы. Критерии делятся на 4 категории: D, C, B и A,
организованные в иерархическом виде, где категория A предназначена для компьютерных систем, обеспечивающих наиболее полную безопасность обрабатываемой информации. В рамках категорий B и C
есть ряд подкатегорий --- классы. Классы также упорядочены в иерархическом виде: системы, относящиеся к категории С и более низким классам категории B, характеризуются набором механизмов
безопасности, которыми они обладают. Гарантия правильного и полного проектирования и внедрения этих систем обеспечивается, главным образом, за счет тестирования тех частей системы, 
которые имеют отношение к безопасности. Имеющие отнощение к безопасности всей системы и ее обеспечивающие части системы в <<Оранжевой книге>> и называются Доверенной Вычислительной Базой (TCB). 
Системы, относящиеся к более высоким классам категории B и к классам категории A, получают атрибуты безопасности исходя непосредственно из своей структуры проектирования и реализации.
Благодаря более тщательному анализу на этапе проектирования системы и достигается уверенность в том, что требуемые функции безопасности корректны, работоспособны и стойки при 
любых обстоятельствах и воздействиях на них.

В рамках каждого категории рассматриваются 4 основные группы критериев: политика безопаности, подотчетность, корректность и документация. Последняя группа описывает типы руководствующих 
документов, необходимых для оцениванния системы в рамках данной категории. Ее расматривать мы не будем. Желающих всё-таки ознакомиться отсылаем к первоисточнику \cite{OrangeBook}.

Рассмотрим критерии оценки надежных компьютерных систем подробнее \cite{OrangeBook}:


\begin{enumerate}
	\item{\textbf{Категория D: минимальная защита}}\\
	Данная категория содержит только один класс компьютерных систем, прошедших оценку, но не отвечающих требованиям более высоких классов оценки.
	\item{\textbf{Категория C: дискреционная защита}}\\
	Классы критериев оценки безопасности в этой категории предусматривают обеспечение компьютерной системой дискреционной защиты и подотчетности субъектов и инициирумеых ими действий.
	\begin{enumerate}
		\item{\textbf{Класс C1: дискреционная защита}}\\
		TCB компьютерных систем, удовлетворяющих критериям оценки безопасности класса C1, номинально удовлетворяет требованиям дискреционной защиты, обеспечивая разделение пользователей и данных.
		Она включает в себя средства контроля доступа, обеспечивающие разграничие доступа пользователей к конфиденциальной информации в индивидуальном порядке и непозволяющие несанкционированное
		изменение информации. Предполагается, что компьютерных системах класса C1 пользователи будут обрабатывать данные одного уровня конфиденциальности. Для систем класса C1 выдвигаются следующие
		минимальные требования:
		\begin{enumerate}
			\item{\textbf{Политика безопаности}}
			\begin{enumerate}
				\item{\textbf{Дискреционный контроль доступа}}\\
				TCB должен определять и контролировать доступ именованных субъектов к именованным объектам (\textit{напр.} файлы и программы) в компьютерной системе. Механизмы обеспечения контроля
				(\textit{напр.} self/group/other контроль, ACL) должны позволять пользователям определять и контролировать общий доступ к объектам именованными субъектами и/или определенной группой
				субъектов.
			\end{enumerate}
			\item{\textbf{Подотчетность}}
			\begin{enumerate}
				\item{\textbf{Идентификация и аутентификация}}\\
				TCB должен требовать идентификации пользователей перед началом выполнения любых операций, посредником в которых является. Кроме того, TCB должен использовать защищенные механизмы
				аутентификации личности пользователей. TCB также должен защищать данные аутентификации от несанкционированного доступа.
			\end{enumerate}
			\item{\textbf{Гарантия (Корректность)}}
			\begin{enumerate}
				\item{\textbf{Гарантия работы}}
				\begin{itemize}
					\item{\textbf{Aрхитектура системы}}\\
					TCB должен поддерживать рабочую вычислительную среду для собственного исполнения, защищать ее от внешнего вмешательства и изменения. Ресурсы, контролируемые TCB, могут быть
					некоторым подмножеством объектов и субъектов в компьютерной системе.
					\item{\textbf{Целостность системы}}\\
					Должны быть предусмотрены аппаратные и/или программные средства, могут быть использованы для периодической проверки правильности работы аппаратных и программных элементов TCB.
				\end{itemize}
				\item{\textbf{Гарантия жизненного цикла}}
				\begin{itemize}
					\item{\textbf{Тестирование на безопасность}}\\
					Защитные механизмы должны быть протестированы и проверены на работоспособность в соответствии с сопроводителной документацией на компьютерную систему.
					Тестирование проводится с целью удостоверения в отсутствии способов обхода неавторизованным пользователем защитных механизмов TCB.  
				\end{itemize}
			\end{enumerate}
		\end{enumerate}
		\item{\textbf{Класс C2: защита контролируемого доступа}}\\
		Компьютерные системы, относящиеся к этому классу, реализуют более <<тонкий>> дискреционный контроль доступа, чем системы класса C1, заставляя пользователей нести ответсвенность за свои
		действия с помощью процедур входа в систему, аудита событий, имеющих отношение к безопасности, и изоляции ресурсов. Для систем класса C2 выдвигаются следующие минимальные требования:
		\begin{enumerate}
			\item{\textbf{Политика безопаности}}
			\begin{enumerate}
				\item{\textbf{Дискреционный контроль доступа}}\\
				TCB должен определять и контролировать доступ именованных субъектов к именованным объектам (\textit{напр.} файлы и программы) в компьютерной системе. Механизмы обеспечения контроля
				(\textit{напр.} self/group/other контроль, ACL) должны позволять пользователям определять и контролировать общий доступ к объектам именованными субъектами и/или определенной группой
				субъектов и обеспечивать контроль ограничения распространения прав доступа. Механизм дискреционного контроля доступа должен либо путем явного действия пользователя, либо по умолчанию
				обеспечивать защиту объектов от несанкционированного доступа. Средства управления доступом должны позволять включать или исключать доступ к объектам с точностью до отдельного
				пользователя. Право на доступ к объекту пользователям, не имеющим данного права, должно назначаться только авторизованным пользователем. 
				\item{\textbf{Переиспользование объектов}}\\
				Все разрешения на доступ к информации, содержащейся в объекте хранения, должны быть аннулированы до момента первоначального назначения, распределения или перераспределения объекта
				субъекту из пула неиспользованных объектов хранения TCB. Никакая информация, в том числе и зашифрованная, созданная в результате действий другого субъекта, не должна быть доступна
				другому субъекту, получившему доступ к объекту хранения, ранее возвращенный системе.  	
			\end{enumerate}
			\item{\textbf{Подотчетность}}
			\begin{enumerate}
				\item{\textbf{Идентификация и аутентификация}}\\
				TCB должен требовать идентификации пользователей перед началом выполнения любых операций, посредником в которых является. Кроме того, TCB должен использовать защищенные механизмы
				аутентификации личности пользователей. TCB также должен защищать данные аутентификации от несанкционированного доступа. TCB должен быть способен обеспечить подотчетность отдельного
				пользователя, предоставляя возможность идентификации каждого пользователя. TCB должен также обеспечивать соответствие пользователя с совершаемыми им действиями, подлежащими аудиту.
				\item{\textbf{Аудит}}\\
				TCB должен быть способен создавать, поддерживать и защищать от модификации и несанкционированного доступа или удаления журнал аудита доступов к защищаемым объектам. Журнал аудита 
				должен быть защищен средтсвами TCB таким образом, чтобы доступ на чтение был ограничен только уполномоченными на чтение журнала аудита. TCB должен регистрировать следующие события
				в системе: использование механизмов идентификации и аутентификации, передача объектов в пользовательское адресное пространство (\textit{напр.} открытие файла, запуск программы),
				удаление объектов, действия операторов и системных администраторов и/или сотрудников безопасности системы и других событий, связанных с безопасностью. Для каждого зарегистрированного
				события в журнале аудита должно быть указано: время и дата наступления события, пользователь, совершивший действие, тип события, результат события (успех/неудача). Для событий
				идентификации/аутентификации в журнале аудита должно быть указано происхождение запроса (\textit{прим. } кем инициировано событие) (\textit{напр.} идентификатор терминала). Для событий,
				вводящих объект в адресное пространство пользователя и удлаяющих объект, запись в журнале ауидта должна содержать имя объекта. Администратор системы должен иметь возможность выборочно
				проверять действия любого пользователя системы.
			\end{enumerate}
			\item{\textbf{Гарантия (Корректность)}}
			\begin{enumerate}
				\item{\textbf{Гарантия работы}}
				\begin{itemize}
					\item{\textbf{Aрхитектура системы}}\\
					TCB должен поддерживать рабочую вычислительную среду для собственного исполнения, защищать ее от внешнего вмешательства и изменения. Ресурсы, контролируемые TCB, могут быть
					некоторым подмножеством объектов и субъектов в компьютерной системе. TCB должен изолировать защищаемые ресурсы таким образом, чтобы на них распространялись требования контроля
					доступа и аудита.
					\item{\textbf{Целостность системы}}\\
					Должны быть предусмотрены аппаратные и/или программные средства, могут быть использованы для периодической проверки правильности работы аппаратных и программных элементов TCB.
				\end{itemize}
				\item{\textbf{Гарантия жизненного цикла}}
				\begin{itemize}
					\item{\textbf{Тестирование на безопасность}}\\
					Защитные механизмы должны быть протестированы и проверены на работоспособность в соответствии с сопроводителной документацией на компьютерную систему.
					Тестирование проводится с целью удостоверения в отсутствии способов обхода неавторизованным пользователем защитных механизмов TCB. Тестирование должно также включать в себя
					поиск уязвимостей, позволяющих нарушить изоляцию ресурсов или получить несанкционированный доступ к журналам аудита или к данным аутентификации. 
				\end{itemize}
			\end{enumerate}
		\end{enumerate}
	\end{enumerate}
	\item{\textbf{Категория B: мандатная защита}}\\
	Понятие TCB, сохраняющей целостность меток конфиденциальности и используещей их для обеспечения мандатного контроля доступа, является основным требованием в этой категории. Системы, попадающие
	под эту категорию, должны иметь метки конфиденциальности для основных структур данных в системе. Разработчик системы также предоставляет модель политики безопаности, на которой основана TCB, и
	саму спецификацию TCB. Должны быть предоставлены доказательства, демонстрирующие, что концепция монитора ссылок была реализована в системе.  	 
	\begin{enumerate}
		\item{\textbf{Класс B1: защита метками безопаности}}\\
		B1 класс требует всех функций, необходимых для класса C2. В дополнение к ним, в системе должны присутствовать неформальное утверждение модели политики безопасности, маркировка данным метками,
		мандатный контроль доступа к именованным объектам и субъектам. Должна присутствовать возможность точечной маркировки экспортируемой информации. Все уязвимости, выявленные во время тестирования,
		должны быть устранены. Для систем класса C2 выдвигаются следующие минимальные требования:
		\item{\textbf{Политика безопаности}}
		\begin{enumerate}
			\item{\textbf{Дискреционный контроль доступа}}\\
			TCB должен определять и контролировать доступ именованных субъектов к именованным объектам (\textit{напр.} файлы и программы) в компьютерной системе. Механизмы обеспечения контроля
			(\textit{напр.} self/group/other контроль, ACL) должны позволять пользователям определять и контролировать общий доступ к объектам именованными субъектами и/или определенной группой
			субъектов и обеспечивать контроль ограничения распространения прав доступа. Механизм дискреционного контроля доступа должен либо путем явного действия пользователя, либо по умолчанию
			обеспечивать защиту объектов от несанкционированного доступа. Средства управления доступом должны позволять включать или исключать доступ к объектам с точностью до отдельного
			пользователя. Право на доступ к объекту пользователям, не имеющим данного права, должно назначаться только авторизованным пользователем.
			\item{\textbf{Переиспользование объектов}}\\
			Все разрешения на доступ к информации, содержащейся в объекте хранения, должны быть аннулированы до момента первоначального назначения, распределения или перераспределения объекта
			субъекту из пула неиспользованных объектов хранения TCB. Никакая информация, в том числе и зашифрованная, созданная в результате действий другого субъекта, не должна быть доступна
			другому субъекту, получившему доступ к объекту хранения, ранее возвращенный системе.
			\item{\textbf{Метки}}\\
			TCB должен поддерживать метки конфиденциальности, связанные с каждым субъектом и объектом хранения, находящимся под контролем (\textit{напр.} файл, устройство, процесс). Эти метки должны
			быть основой принятия решений в рамках мандатного контроля доступа. При импорте не маркированных данных TCB должен запросить и получить от уполномоченного пользователя уровень
			конфиденциальности данных, и все такие действия должны заносится в журнал аудита.
			\begin{itemize}
				\item{\textbf{Целостность меток}}\\
				Метки конфиденциальности должны точно отражать уровень конфиденциальности конкретных субъектов и объектов, с которыми они ассоциированы. При экспорте из TCB метки конфиденциальности
				должны однозначно соответствовать внутренним меткам и быть ассоциированы с экспортируемой информацией.
				\item{\textbf{Экспорт маркированной информации}}\\
				TCB должен классифицировать каждый канал связи и устройство ввода/вывода как одноуровневый или многоуровневый (\textit{прим. } передается ли метка конфиденциальности вместе с данными
				или нет). Любое изменение этой классификации должно производиться вручную и проверяться TCB. TCB должен поддерживать и иметь возможность проверять любые изменения в уровне
				конфиденциальности или классификации, связанной с каналом связи или устройством ввода/вывода.   
			\end{itemize}
			\item{\textbf{Мандатный контроль доступа}}\\
			TCB должен применять мандатный контроль доступа ко всем субъектам и объектам хранения, находящимся под контролем (\textit{напр. } файл, устройство, процесс). Этим субъектам и объектам
			должны быть присвоены метки конфиденциальности, представляющие собой комбинацию иерархических уровней классификации и неиерархических категорий, и эти метки должны быть основой принятия 
			решений в рамках мандатного контроля доступа. TCB должен поддерживать два и более уровня конфиденциальности. Для всех доступов субъектов к объектам, находящимся под контролем  TCB, должны
			должны выполняться следующие требования: субъект может читать объект только если иерархическая классификация уровня конфиденциальности субъекта больше или равна иерархической классификации
			уровня конфиденциальности объекта, а неиерархические категории уровня конфиденциальности субъекта включают в себя все неиерархические категории уровня конфиденциальности объекта. Субъект может
			писать в объект только если иерархическая классификация уровня конфиденциальности субъекта меньше или равна иерархической классификации уровня конфиденциальности объекта, и все 
			неиерархические категории уровня конфиденциальности субъекта включены в неиерархические категории уровня конфиденциальности объекта. Идентификационные и аутентификационные данные должны 
			использоваться TCB для удостоверения личности пользователя и для обеспечения того, чтобы уровень конфиденциальности и авторизация субъектов, внешних для TCB, которые могут быть созданы для
			деятельности от имени отдельного пользователя, определялись уровнем доступа и авторизацией этого пользователя.
		\end{enumerate}
		\item{\textbf{Подотчетность}}
		\begin{enumerate}
			\item{\textbf{Идентификация и аутентификация}}\\
			TCB должен требовать идентификации пользователей перед началом выполнения любых операций, посредником в которых является. Кроме того, TCB должен хранить данные аутентификации, 
			которые включают в себя информацию для проверки личности отдельных пользователей, а также информацию для определения допуска и полномочий отдельных пользователей. Эти данные должны 
			использоваться TCB для аутентификации личности пользователя и обеспечения того, что уровень конфиденциальности и полномочия внешних по отношению к TCB субъектов, которые 
			могут быть созданы для действий от имени отдельного пользователя, определяются допуском и полномочиями этого пользователя. TCB должен защищать данные аутентификации от 
			несанкционированного доступа неавторизованным пользователь. TCB должен обеспечивать подотчетность отдельного пользователя, предоставляя возможность его уникальной идентификации.
			TCB должен также обеспечивать соответствие пользователя с совершаемыми им действиями, подлежащими аудиту.
			\item{\textbf{Аудит}}\\
			TCB должен быть способен создавать, поддерживать и защищать от модификации и несанкционированного доступа или удаления журнал аудита доступов к защищаемым объектам. Журнал аудита 
			должен быть защищен средтсвами TCB таким образом, чтобы доступ на чтение был ограничен только уполномоченными на чтение журнала аудита. TCB должен регистрировать следующие события
			в системе: использование механизмов идентификации и аутентификации, передача объектов в пользовательское адресное пространство (\textit{напр.} открытие файла, запуск программы),
			удаление объектов, действия операторов и системных администраторов и/или сотрудников безопасности системы и других событий, связанных с безопасностью. Для каждого зарегистрированного
			события в журнале аудита должно быть указано: время и дата наступления события, пользователь, совершивший действие, тип события, результат события (успех/неудача). Для событий
			идентификации/аутентификации в журнале аудита должно быть указано происхождение запроса (\textit{прим. } кем инициировано событие) (\textit{напр.} идентификатор терминала). Для событий,
			вводящих объект в адресное пространство пользователя и удлаяющих объект, запись в журнале ауидта должна содержать имя объекта. Администратор системы должен иметь возможность выборочно
			проверять действия любого пользователя системы.
		\end{enumerate}
		\item{\textbf{Гарантия (Корректность)}}
		\begin{enumerate}
			\item{\textbf{Гарантия работы}}
			\begin{itemize}
				\item{\textbf{Aрхитектура системы}}\\
				TCB должен поддерживать рабочую вычислительную среду для собственного исполнения, защищать ее от внешнего вмешательства и изменения. Ресурсы, контролируемые TCB, могут быть
				некоторым подмножеством объектов и субъектов в компьютерной системе. TCB должен поддерживать изоляцию процессов путем предоставления отдельных адресных пространств под его контролем.
				TCB должен изолировать защищаемые ресурсы таким образом, чтобы на них распространялись требования контроля доступа и аудита.
				\item{\textbf{Целостность системы}}\\
				Должны быть предусмотрены аппаратные и/или программные средства, могут быть использованы для периодической проверки правильности работы аппаратных и программных элементов TCB.
			\end{itemize}
			\item{\textbf{Гарантия жизненного цикла}}
			\begin{itemize}
				\item{\textbf{Тестирование на безопасность}}\\
				Защитные механизмы должны быть протестированы и проверены на работоспособность в соответствии с сопроводителной документацией на компьютерную систему.  Команда специалистов,
				досконально разбирающихся в специфике реализации TCB, должна провести тщательный анализ и тестирование проектной документации, исходного и объектного кода системы. Целями такой
				проверки должны быть: выявление недостатков проектирования и реализации, которые позволят субъекту, внешнему по отношению к TCB, совершать действия (чтение, изменение или удаление данных),
				обычно отклоняемые мандатными и дискреционными политиками безопасности, применяемыми в TCB; а также проверка того, что ни один субъект не способен привести (без разрешения на это)
				TCB в состояние, при котором он не сможет отвечать на запросы от других пользователей системы. Все обнаруженные недостатки должны быть устранены, а TCB повторно протестирован для 
				демонстрации отсутствия старых и новых недостатков. 
				\item{\textbf{Спецификация и верификация дизайна системы}}\\
				Неформальная или формальная модель политики безопасности, поддерживаемая TCB, должна поддерживаться на протяжении всего жизненного цикла системы и соответствовать своим аксиомам.
			\end{itemize}
		\end{enumerate}
		\item{\textbf{Класс B2: защита метками безопаности}}\\
	\end{enumerate}
\end{enumerate}


\subsubsection{Понятие политики безопасности}
Согласно \cite{GOST50922}: политика безопасности --- совокупность документированных
правил, процедур, практических приемов или руководящих принципов в области безопасности
информации, которыми руководствуется организация в своей деятельности (то есть как организация
обрабатывает, защищает и распространяет информацию). Причём, политика безопасности относится
к активным методам защиты, поскольку учитывает анализ возможных угроз и выбор адекватных мер
противодействия.

\subsubsection{Совместное применение различных политик безопасности в рамках единой модели}
Проблема совмещения различных политик безопасности возникает достаточно часто при администрировании компьютерных систем. Стандарты защиты информации в автоматизированных системах подразумевают наличие более одной политики разграничения доступа.

Так, в <<Оранжевой книге>> использование только дискреционного разделения доступа относит компьютерную систему к одному из классов безопасности группы <<C>>, тогда как добавление мандатного контроля доступа позволяет претендовать на более высокий класс защищенности группы <<B>>. Причем <<Оранжевой книгой>> подразумевается именно добавление мандатной политики безопасности (МПБ) с сохранением возможностей дискреционной политики безопасности (ДПБ).

В качестве еще одного примера совмещения политик безопасности можно привести системы управления базами данных, функционирующие на базе операционных систем семейства Windows. В системах управления базами данных наиболее распространенной является ролевая политика безопасности, но при этом данные хранятся в файлах, доступ к которым разграничивается операционной системой. В операционных системах базовой является ДПБ, но при этом реализуется на определенном уровне МПБ. Таким образом, требуется сопряжение трех различных политик безопасности. \cite{CombinedSecurityPolicies}

\subsubsection{Интерпретация TCSEC для надежных СУБД (TDI)}
В дополнение к <<Оранжевой книге>> TCSEC, регламентирующей вопросы обеспечения безопасности в компьютерных системах, существуют аналогичный документ Национального центра компьютерной безопасности США для СУБД --- Trusted Database Management System Interpretation (TDI),--- так называемая <<Пурпурная книга >> \cite{PurpleBook}.
В ней приведены интерпретации требований TCSEC применительно к СУБД.

Рассмотрим эти интерпретации подробнее \cite{PurpleBook}:
\begin{enumerate}
	\item \textbf{Политика безопасности}
	\begin{itemize}
		\item \textbf{Дискреционный контроль доступа}\\
		Требование для дискреционного контроля доступа не изменяется в данном документе и применяется так, как указано в TCSEC.
		\item \textbf{Повторное использование объекта}\\
		Требование для повторного использования объекта не изменяется в данном документе и применяется так, как указано в TCSEC.
		\item \textbf{Метки безопаности}\\
		Требования к меткам в TCSEC полностью применимы и к СУБД. Основное различие между требованиями TCSEC к меткам, применяемым к ОС и СУБД, заключается в том, какие объекты хранения помечены, а не как обрабатываются эти метки.
		Начиная с класса B1, СУБД должны связывать метки со всеми объектами хранения, находящимися под их контролем. Степень детализации защищаемых объектов хранения определяется разработчиком СУБД.
		\item \textbf{Мандатный контроль доступа}\\
		Требование для мандатного контроля доступа не изменяется в данном документе и применяется так, как указано в TCSEC. Однако у этого требования есть несколько тонких последствий, о которых разработчику следует знать. Более подробно можно узнать в \cite{PurpleBook} Appendix B, item 8.
	\end{itemize}
	\item \textbf{Подотчетность}
	\begin{itemize}
		\item \textbf{Идентификация и аутентификация}\\
		Требование для идентификации и аутентификации не изменяется в данном документе и применяется так, как указано в TCSEC.
		\item \textbf{Аудит}\\
		Требования к аудиту в TCSEC применяются также и к СУБД. TCB должен быть способен вести аудит доступов и попыток доступа к объектам, находящимся под защитой мандатной и дискреционной политик безопасности.
		% -------------------------------------------------
		% TODO: fill out the list according to the Purple Book
		% also check https://www.jetinfo.ru/informaczionnaya-bezopasnost-obzor-osnovnyh-polozhenij-chast-1/
		% and https://dorlov.blogspot.com/2010/01/issp-03-8.html
		% -------------------------------------------------
	\end{itemize}
\end{enumerate}

\subsubsection{Оценка надежности СУБД как компоненты вычислительной системы}
    % Seems to be Appendix B from \cite{PurpleBook}
\subsubsection{Монитор ссылок}
Монитор обращений (reference monitor) – это абстрактная машина, являющаяся посредником, через которого проходят все обращения субъектов к объектам. Монитор обращений проверяет, что субъекты имеют необходимые права доступа, защищая объекты от несанкционированного доступа и изменения.
Чтобы система достигла высокого уровня доверия, необходимо, чтобы субъекты (программы, пользователи и процессы) были полностью авторизованы перед их доступом к объектам (файлам, программам и ресурсам).
Субъекту не должно быть позволено использовать запрошенный ресурс, пока ему не будут предоставлены соответствующие привилегии доступа. Монитор обращений – это концепция управления доступом, а не реальный физический компонент, поэтому его часто называют «концепцией монитора обращений» или «абстрактной машиной». \cite{CISSP}

Ядро безопасности (security kernel) состоит из компонентов аппаратного обеспечения, программного обеспечения и прошивок, попадающих в рамки ТСВ и реализующих концепцию монитора обращений.
Ядро безопасности осуществляет посредничество при доступе и работе субъектов с объектами. Ядро безопасности – это ядро ТСВ, это наиболее часто используемый подход к построению доверенных компьютерных систем (trusted computing system). \cite{CISSP}

Есть три основных требования к ядру безопасности:
\begin{itemize}
	\item Изолированность (неотслеживаемость работы)
	\item Полнота (невозможность обойти)
	\item Верифицируемость (возможность анализа и тестирования)
\end{itemize}

\subsubsection{Применение TCSEC к СУБД непосредственно}
%
%	what????????
%
\subsubsection{Элементы СУБД, к которым применяются TDI: метки, аудит, архитектура системы, спецификация, верификация, проектная документация}

\cite{PurpleBook} IR-3 -- IR-6

\subsubsection{Критерии безопасности ГТК/ФСТЭК}

Некоторое время ФСТЭК (в прошлом ГТК, до 2004 г.) использовал критерии безопасности баз данных, заданные документом <<Безопасность информационных технологий. Критерии оценки безопасности информационных технологий>> от 19.06.02 г. № 187. Этот документ был разработан в ГТК и сейчас выведен из употребления. \cite{GTKFSTEK}

Однако, в настоящее время (на 2023 год), ФСТЭК работает по еще более старому документу, которым заменили более новый. Не ищите тут логику. Называется он так:
<<Средства вычислительной техники. Защита от несанкционированного доступа к информации. Показатели защищенности от несанкционированного доступа к информации>>, утвержденного решением Государственной технической комиссии при Президенте Российской Федерации от 30 марта 1992 г.

\href{https://fstec.ru/component/attachments/download/293}{ФСТЭК 187 (устаревший)}

\href{https://fstec.ru/component/attachments/download/296}{ФСТЭК Средства вычислительной техники. Защита от несанкционированного доступа к информации. Показатели защищенности от несанкционированного доступа к информации (актуальный)}

Всего существует семь классов СВТ (средств вычислительной техники), в них входит что угодно, СВТ - это совокупность ПО и аппаратной части в любых сочетаниях. Шесть классов <<рабочие>> и седьмой <<мусорный>>, для тех, которые не дотягивают даже до шестого.

Классы дополнительно делятся на четыре группы, отличающиеся качественным уровнем защиты:
\begin{itemize}
	\item Первая группа содержит только один седьмой класс;
	\item Вторая группа характеризуется дискреционной защитой и содержит шестой и пятый классы;
	\item Третья группа характеризуется мандатной защитой и содержит четвертый, третий и второй классы;
	\item Четвертая группа характеризуется верифицированной защитой и содержит только первый класс.
\end{itemize}

Сами требования весьма базовые и к функционированию непосредственно баз данных относятся опосредованно. Но, если в вашей БД будет лежать что-то ценное, вы автоматом попадаете под раздачу, так что эти требования надо выполнять.

Оценка класса защищенности СВТ проводится в соответствии с
<<Положением о сертификации средств и систем вычислительной техники и связи по требованиям защиты информации>> и <<Временным положением по организации разработки, изготовления и эксплуатации программных и технических средств защиты информации от несанкционированного доступа в автоматизированных системах и средствах вычислительной техники>> и другими документами.

Le meme в том, что первый документ на сайте ФСТЭК отсутствует и видимо был заменен на <<Положение о системе сертификации средств защиты информации>>. Куда делись требования ко всему остальному не ясно, это еще предстоит выяснить. Второй документ (тоже образца 92 года) на сайте есть в чистом виде и все еще актуален.

\href {https://fstec.ru/component/attachments/download/1883}{Положение о системе сертификации средств защиты информации}

\href {https://fstec.ru/component/attachments/download/315}{Временное положение по организации разработки, изготовления и эксплуатации программных и технических средств защиты информации от несанкционированного доступа в автоматизированных системах и средствах вычислительной техники}

Не уверены в том, что юзать? Спросите старших товарищей на работе или напишите на горячую линию ФСТЭК-а, они там для этого и сидят.

\href{https://fstec.ru/kontakty}{Контакты ФСТЭК}

В соответствии c все тем же документом образца 92 года, выбор класса защищенности СВТ для автоматизированных систем, создаваемых на базе защищенных СВТ, зависит от грифа секретности обрабатываемой в АС информации, условий эксплуатации и расположения объектов системы.

Эти документы регулируют в первую очередь разработку продукта. Защита систем в целом подчиняется приказу 239 (меры ИБ, как защищать) и Методике оценки угроз безопасности информации от 2021 года, еще актуальной на 2023 год (как правильно бояться).

Итак, в документе <<Средства вычислительной техники. Защита от несанкционированного доступа к информации. Показатели защищенности от несанкционированного доступа к информации>> вы найдете список вещей которые надо держать в уме и требования по их реализации. Классификации нет, механизмы подлежащие реализации смешаны с требованиями к комплекту документации.
Список следующий:
\begin{itemize}
	\item Дискреционный принцип контроля доступа
	\item Мандатный принцип контроля доступа
	\item Очистка памяти
	\item Изоляция модулей
	\item Маркировка документов
	\item Защита ввода и вывода на отчуждаемый физический носитель информации
	\item Сопоставление пользователя с устройством
	\item Идентификация и аутентификация
	\item Гарантии проектирования
	\item Регистрация
	\item Взаимодействие пользователя с комплексом средств защиты
	\item Надежное восстановление
	\item Целостность комплекса средств защиты
	\item Контроль модификации
	\item Контроль дистрибуции
	\item Гарантии архитектуры
	\item Тестирование
	\item Руководство для пользователя
	\item Руководство по комплексу средств защиты
	\item Тестовая документация
	\item Конструкторская (проектная) документация
\end{itemize}

Сертификация ФСТЭК – процедура получения документа, подтверждающего, что средство защиты информации соответствует требованиям нормативных и методических документов ФСТЭК России.

Сертификация ФСТЭК для средств защиты информации создана для того, чтобы обеспечить:
\begin{itemize}
    \item Защиту конфиденциальной информации строго определенного уровня.
    \item Возможность для потребителей выбирать качественные и эффективные средства защиты информации.
    \item Содействие формированию рынка защищенных информационных технологий и средств их обеспечения.
\end{itemize}

Сертификация ФСТЭК необходима для строго определенных сфер деятельности (примерно Всех важных и делающих деньги, кроме закладок, вебкама и геймдизайна, и то не факт) и ее необходимость зависит от того, какая именно информация будет обрабатываться в той или иной информационной системе.

Сферы деятельности с обязательной сертификацией средств защиты информации:
\begin{itemize}
    \item Государственные информационные системы (ГИС).
    \item Автоматизированные системы управления технологическим процессом (АСУ ТП).
    \item Персональные данные (ЗПДн).
    \item Критическая информационная инфраструктура (КИИ). Их тоже целый перечень, см. ниже.
    \item Государственная тайна (ЗГТ).
    \item Конфиденциальная информация (служебная информация).
\end{itemize}

Список сфер КИИ:
\begin{itemize}
    \item Здравоохранение;
    \item Наука;
    \item Транспорт;
    \item Связь;
    \item Энергетика;
    \item Банковская и иные сферы финансового рынка;
    \item Топливно-энергетический комплекс;
    \item Атомная энергия;
    \item Оборонная и ракетно-космическая промышленность;
    \item Горнодобывающая, металлургическая и химическая промышленность.
\end{itemize}

Несертифицированный продукт невозможно использовать в тех сферах деятельности, где обязательно использование сертифицированных продуктов.

Использование компанией несертифицированной продукции в сфере деятельности, требующей обязательной сертификации средств защиты информации, может повлечь за собой серьезные последствия: от больших штрафов до уголовной ответственности для руководителей.

В пределах одной компании может быть несколько информационных систем. Некоторые из них нуждаются в сертифицированных средствах защиты, а некоторые – нет. Все зависит от характера информации, которая обрабатывается в этих системах. В случае, если данные касаются вышеперечисленных сфер деятельности, сертификация обязательна.

ФСТЭК контролирует практически все области защитф информации, исключениями явлюяются линии связи (их контролирует Минскомсвязи), и средства криптографической защиты (их контролирует ФСБ).

Ситуация с шифрованием очень зыбкая, какое-то время ФСБ наступала даже на антивирусы, поскольку обновления в них защищаются в том числе криптографическими методами. Но, процесс этот к счастью заглох. Требования ФСБ - тайна покрытая мраком и процедура сертификации у них крайне непрозрачная, осуществляют ее всего несколько особо доверенных фирм. Если криптографические функции вашего продукта не являются его основным назначением, то сертификация от ФСБ необязательна, для выхода на рынок вам хватит сертификата ФСТЭК.

Ваша дорога из желтого кирпича изложена в двух вышеупомянутых документах.

<<Полжение о системе сертификации средств защиты информации>> - это просто инструкция, описывающая процесс сертификации и ответсвенность сторон.

Положение определяет состав участников системы сертификации средств защиты информации, создаваемой ФСТЭК России, а также организацию и порядок сертификации продукции, используемой в целях защиты сведений, составляющих государственную тайну или относимых к охраняемой в соответствии с законодательством Российской Федерации иной информации ограниченного доступа, являющейся государственным информационным ресурсом и (или) персональными данными, продукции, сведения о которой составляют государственную тайну, подлежащей сертификации в рамках указанной системы.

Список участников такой:
\begin{itemize}
    \item федеральный орган по сертификации;
    \item организации, аккредитованные ФСТЭК России в качестве органа по сертификации (органы по сертификации);
    \item организации, аккредитованные ФСТЭК России в качестве испытательной лаборатории (далее - испытательные лаборатории);
    \item изготовители средств защиты информации (вы находитесь вот тут)
\end{itemize}
Окей, что такое <<временное положение?>>. Ну, одно точно, нет ничего постояннее временного.

Положение устанавливает единый на территории Российской Федерации порядок исследований и разработок в области:
\begin{itemize}
	\item Защиты информации, обрабатываемой автоматизированными системами различного уровня и назначения, от несанкционированного доступа;
	\item Создания средств вычислительной техники общего и специального назначения, защищенных от утечки, искажения или уничтожения информации за счет НСД, в том числе программных и технических средств защиты информации от НСД;
	\item Создания программных и технических средств защиты информации от НСД в составе систем защиты секретной информации в создаваемых АС.
\end{itemize}
Положение определяет следующие основные вопросы:
\begin{itemize}
	\item Организационную структуру и порядок проведения работ по защите информации от НСД и взаимодействия при этом на государственном уровне;
	\item Систему государственных нормативных актов, стандартов, руководящих документов и требований по этой проблеме;
	\item Порядок разработки и приемки защищенных СВТ, в том числе программных и технических (в частности, криптографических) средств и систем защиты информации от НСД;
	\item Порядок приемки указанных средств и систем перед сдачей в эксплуатацию в составе АС, порядок их эксплуатации и контроля за работоспособностью этих средств и систем в процессе эксплуатации.
\end{itemize}

Вот так, как-то. Добро пожаловать в увлекательный мир нормативно-правовой документации в области защиты информации.
