\subsection{Критерии защищенности БД}

\subsubsection{Критерии оценки надежных компьютерных систем (TCSEC)}
<<Критерии безопасности компьютерных систем>> (Trusted Computer System Evaluation Criteria),
получившие неформальное название <<Оранжевая книга>>, были разработаны Министерством обороны
США в 1983 году с целью определения требований безопасности, предъявляемых к аппаратному,
программному и специальному обеспечению компьютерных систем и выработки соответствующей
методологии и технологии анализа степени поддержки политики безопасности в компьютерных
системах военного назначения. В данном документе были впервые нормативно определены такие
понятия, как <<политика безопасности>>, <<ядро безопасности>> (ТСВ) и т.д.

Предложенные в этом документе концепции защиты и набор функциональных требований послужили
основой для формирования всех появившихся впоследствии стандартов безопасности.

В <<Оранжевой книге>> предложены три категории требований безопасности -- политика безопасности,
аудит и корректность, в рамках которых сформулированы шесть базовых требований безопасности.
Первые четыре требования направлены непосредственно на обеспечение безопасности информации,
а два последних -- на качество самих средств защиты.

Рассмотрим эти требования подробнее:

\begin{enumerate}
	\item Политика безопасности
	\begin{itemize}
		\item \textbf{Политика безопасности}
		Система должна поддерживать точно определённую политику безопасности. Возможность осуществления субъектами доступа к объектам должна определяться на основе их идентификации и набора правил управления доступом. Там, где необходимо, должна использоваться политика нормативного управления доступом, позволяющая эффективно реализовать разграничение доступа к категорированной информации (информации, отмеченной грифом секретности — типа <<секретно>>, <<сов. секретно>> и т.д.).
		\item \textbf{Метки} С объектами должны быть ассоциированы метки безопасности, используемые в качестве атрибутов контроля доступа. Для реализации нормативного управления доступом система должна обеспечивать возможность присваивать каждому объекту метку или набор атрибутов, определяющих  степень конфиденциальности (гриф секретности) объекта и/или режимы доступа к этому объекту.
	\end{itemize}
	\item Аудит
	\begin{itemize}
		\item \textbf{Идентификация и аутентификация} Все субъекты должен иметь уникальные идентификаторы. Контроль доступа должен осуществляться на основании результатов идентификации субъекта и объекта доступа, подтверждения подлинности их идентификаторов (аутентификации) и правил разграничения доступа. Данные, используемые для идентификации и аутентификации, должны быть защищены от несанкционированного доступа, модификации и уничтожения и должны быть ассоциированы со всеми активными компонентами компьютерной системы, функционирование которых критично с точки зрения безопасности.
		\item \textbf{Регистрация и учет} Для определения степени ответственности пользователей за действия в системе, все происходящие в ней события, имеющие значение с точки зрения безопасности, должны отслеживаться и регистрироваться в защищенном протоколе. Система регистрации должна осуществлять анализ общего потока событий и выделять из него только те события, которые оказывают влияние на безопасность для сокращения объема протокола и повышения эффективность его анализа. Протокол событий должен быть надежно защищен от несанкционированного доступа, модификации и уничтожения.
	\end{itemize}
	\item Корректность
	\begin{itemize}
		\item \textbf{Контроль корректности} Средства защиты должны содержать независимые аппаратные и~/~или программные компоненты, обеспечивающие работоспособность функций защиты. Это означает, что все средства защиты, обеспечивающие политику безопасности, управление атрибутами и метками безопасности, идентификацию и аутентификацию, регистрацию и учёт, должны находиться под контролем средств, проверяющих корректность их функционирования. Основной принцип контроля корректности состоит в том, что средства контроля должны быть полностью независимы от средств защиты.
		\item \textbf{Непрерывность защиты} Все средства защиты (в т.ч. и реализующие данное требование) должны быть защищены от несанкционированного вмешательства и/или отключения, причем эта защита должна быть постоянной и непрерывной в любом режиме функционирования системы защиты и компьютерной системы в целом. Данное требование распространяется на весь жизненный цикл компьютерной системы. Кроме того, его выполнение является одним из ключевых аспектов формального доказательства безопасности системы.
	\end{itemize}
\end{enumerate}

\subsubsection{Понятие политики безопасности}
Согласно \cite{GOST50922}: политика безопасности -- совокупность документированных
правил, процедур, практических приемов или руководящих принципов в области безопасности
информации, которыми руководствуется организация в своей деятельности (то есть как организация
обрабатывает, защищает и распространяет информацию). Причём, политика безопасности относится
к активным методам защиты, поскольку учитывает анализ возможных угроз и выбор адекватных мер
противодействия.

\subsubsection{Совместное применение различных политик безопасности в рамках единой модели}
Проблема совмещения различных политик безопасности возникает достаточно часто при администрировании компьютерных систем. Стандарты защиты информации в автоматизированных системах подразумевают наличие более одной политики разграничения доступа.

Так, в <<Оранжевой книге>> использование только дискреционного разделения доступа относит компьютерную систему к одному из классов безопасности группы <<C>>, тогда как добавление мандатного контроля доступа позволяет претендовать на более высокий класс защищенности группы <<B>>. Причем <<Оранжевой книгой>> подразумевается именно добавление мандатной политики безопасности (МПБ) с сохранением возможностей дискреционной политики безопасности (ДПБ).

В качестве еще одного примера совмещения политик безопасности можно привести системы управления базами данных, функционирующие на базе операционных систем семейства Windows. В системах управления базами данных наиболее распространенной является ролевая политика безопасности, но при этом данные хранятся в файлах, доступ к которым разграничивается операционной системой. В операционных системах базовой является ДПБ, но при этом реализуется на определенном уровне МПБ. Таким образом, требуется сопряжение трех различных политик безопасности.

\subsubsection{Интерпретация TCSEC для надежных СУБД (TDI)}
В дополнение к <<Оранжевой книге>> TCSEC, регламентирующей вопросы обеспечения безопасности в ОС, существуют аналогичный документ Национального центра компьютерной безопасности США для СУБД -- TDI, (<<Пурпурная книга >>).

\subsubsection{Оценка надежности СУБД как компоненты вычислительной системы}
\subsubsection{Монитор ссылок}
Контроль за выполнением субъектами (пользователями) определённых операций над объектами, путем проверки допустимости обращения (данного пользователя) к программам и данным разрешенному набору действий.

Обязательные качества для монитора обращений:
\begin{itemize}
	\item Изолированность (неотслеживаемость работы)
	\item Полнота (невозможность обойти)
	\item Верифицируемость (возможность анализа и тестирования)
\end{itemize}

\subsubsection{Применение TCSEC к СУБД непосредственно}

\subsubsection{Элементы СУБД, к которым применяются TDI: метки, аудит, архитектура системы, спецификация, верификация, проектная документация}

\href{https://web.archive.org/web/20160303230445/http://ftp.fas.org/irp/nsa/rainbow/tg021.htm}{Пурпурная книга}

\subsubsection{Критерии безопасности ГТК/ФСТЭК}

Некоторое время ФСТЭК (в прошлом ГТК, до 2004 г.) использовал критерии безопасности баз данных, заданные документом <<Безопасность информационных технологий. Критерии оценки безопасности информационных технологий>> от 19.06.02 г. № 187. Этот документ был разработан в ГТК и сейчас выведен из употребления.

Однако, в настоящее время (на 2023 год), ФСТЭК работает по еще более старому документу, которым заменили более новый. Не ищите тут логику. Называется он так:
<<Средства вычислительной техники. Защита от несанкционированного доступа к информации. Показатели защищенности от несанкционированного доступа к информации>>, утвержденного решением Государственной технической комиссии при Президенте Российской Федерации от 30 марта 1992 г.

\href{https://fstec.ru/component/attachments/download/293}{ФСТЭК 187 (устаревший)}

\href{https://fstec.ru/component/attachments/download/296}{ФСТЭК Средства вычислительной техники. Защита от несанкционированного доступа к информации. Показатели защищенности от несанкционированного доступа к информации (актуальный)}

Всего существует семь классов СВТ (средств вычислительной техники), в них входит что угодно, СВТ - это совокупность ПО и аппаратной части в любых сочетаниях. Шесть классов <<рабочие>> и седьмой <<мусорный>>, для тех, которые не дотягивают даже до шестого.

Классы дополнительно делятся на четыре группы, отличающиеся качественным уровнем защиты:
\begin{itemize}
	\item Первая группа содержит только один седьмой класс;
	\item Вторая группа характеризуется дискреционной защитой и содержит шестой и пятый классы;
	\item Третья группа характеризуется мандатной защитой и содержит четвертый, третий и второй классы;
	\item Четвертая группа характеризуется верифицированной защитой и содержит только первый класс.
\end{itemize}

Сами требования весьма базовые и к функционированию непосредственно баз данных относятся опосредованно. Но, если в вашей БД будет лежать что-то ценное, вы автоматом попадаете под раздачу, так что эти требования надо выполнять.

Оценка класса защищенности СВТ проводится в соответствии с
<<Положением о сертификации средств и систем вычислительной техники и связи по требованиям защиты информации>> и <<Временным положением по организации разработки, изготовления и эксплуатации программных и технических средств защиты информации от несанкционированного доступа в автоматизированных системах и средствах вычислительной техники>> и другими документами.

Le meme в том, что первый документ на сайте ФСТЭК отсутствует и видимо был заменен на <<Положение о системе сертификации средств защиты информации>>. Куда делись требования ко всему остальному не ясно, это еще предстоит выяснить. Второй документ (тоже образца 92 года) на сайте есть в чистом виде и все еще актуален.

\href {https://fstec.ru/component/attachments/download/1883}{Положение о системе сертификации средств защиты информации}

\href {https://fstec.ru/component/attachments/download/315}{Временное положение по организации разработки, изготовления и эксплуатации программных и технических средств защиты информации от несанкционированного доступа в автоматизированных системах и средствах вычислительной техники}

Не уверены в том, что юзать? Спросите старших товарищей на работе или напишите на горячую линию ФСТЭК-а, они там для этого и сидят.

\href{https://fstec.ru/kontakty}{Контакты ФСТЭК}

В соответствии c все тем же документом образца 92 года, выбор класса защищенности СВТ для автоматизированных систем, создаваемых на базе защищенных СВТ, зависит от грифа секретности обрабатываемой в АС информации, условий эксплуатации и расположения объектов системы.

Эти документы регулируют в первую очередь разработку продукта. Защита систем в целом подчиняется приказу 239 (меры ИБ, как защищать) и Методике оценки угроз безопасности информации от 2021 года, еще актуальной на 2023 год (как правильно бояться).

Итак, в документе <<Средства вычислительной техники. Защита от несанкционированного доступа к информации. Показатели защищенности от несанкционированного доступа к информации>> вы найдете список вещей которые надо держать в уме и требования по их реализации. Классификации нет, механизмы подлежащие реализации смешаны с требованиями к комплекту документации.
Список следующий:
\begin{itemize}
	\item Дискреционный принцип контроля доступа
	\item Мандатный принцип контроля доступа
	\item Очистка памяти
	\item Изоляция модулей
	\item Маркировка документов
	\item Защита ввода и вывода на отчуждаемый физический носитель информации
	\item Сопоставление пользователя с устройством
	\item Идентификация и аутентификация
	\item Гарантии проектирования
	\item Регистрация
	\item Взаимодействие пользователя с комплексом средств защиты
	\item Надежное восстановление
	\item Целостность комплекса средств защиты
	\item Контроль модификации
	\item Контроль дистрибуции
	\item Гарантии архитектуры
	\item Тестирование
	\item Руководство для пользователя
	\item Руководство по комплексу средств защиты
	\item Тестовая документация
	\item Конструкторская (проектная) документация
\end{itemize}

Сертификация ФСТЭК – процедура получения документа, подтверждающего, что средство защиты информации соответствует требованиям нормативных и методических документов ФСТЭК России.

Сертификация ФСТЭК для средств защиты информации создана для того, чтобы обеспечить:
\begin{itemize}
    \item Защиту конфиденциальной информации строго определенного уровня.
    \item Возможность для потребителей выбирать качественные и эффективные средства защиты информации.
    \item Содействие формированию рынка защищенных информационных технологий и средств их обеспечения.
\end{itemize}

Сертификация ФСТЭК необходима для строго определенных сфер деятельности (примерно Всех важных и делающих деньги, кроме закладок, вебкама и геймдизайна, и то не факт) и ее необходимость зависит от того, какая именно информация будет обрабатываться в той или иной информационной системе.

Сферы деятельности с обязательной сертификацией средств защиты информации:
\begin{itemize}
    \item Государственные информационные системы (ГИС).
    \item Автоматизированные системы управления технологическим процессом (АСУ ТП).
    \item Персональные данные (ЗПДн).
    \item Критическая информационная инфраструктура (КИИ). Их тоже целый перечень, см. ниже.
    \item Государственная тайна (ЗГТ).
    \item Конфиденциальная информация (служебная информация).
\end{itemize}

Список сфер КИИ:
\begin{itemize}
    \item Здравоохранение;
    \item Наука;
    \item Транспорт;
    \item Связь;
    \item Энергетика;
    \item Банковская и иные сферы финансового рынка;
    \item Топливно-энергетический комплекс;
    \item Атомная энергия;
    \item Оборонная и ракетно-космическая промышленность;
    \item Горнодобывающая, металлургическая и химическая промышленность.
\end{itemize}

Несертифицированный продукт невозможно использовать в тех сферах деятельности, где обязательно использование сертифицированных продуктов.

Использование компанией несертифицированной продукции в сфере деятельности, требующей обязательной сертификации средств защиты информации, может повлечь за собой серьезные последствия: от больших штрафов до уголовной ответственности для руководителей.

В пределах одной компании может быть несколько информационных систем. Некоторые из них нуждаются в сертифицированных средствах защиты, а некоторые – нет. Все зависит от характера информации, которая обрабатывается в этих системах. В случае, если данные касаются вышеперечисленных сфер деятельности, сертификация обязательна.

ФСТЭК контролирует практически все области защитф информации, исключениями явлюяются линии связи (их контролирует Минскомсвязи), и средства криптографической защиты (их контролирует ФСБ).

Ситуация с шифрованием очень зыбкая, какое-то время ФСБ наступала даже на антивирусы, поскольку обновления в них защищаются в том числе криптографическими методами. Но, процесс этот к счастью заглох. Требования ФСБ - тайна покрытая мраком и процедура сертификации у них крайне непрозрачная, осуществляют ее всего несколько особо доверенных фирм. Если криптографические функции вашего продукта не являются его основным назначением, то сертификация от ФСБ необязательна, для выхода на рынок вам хватит сертификата ФСТЭК.

Ваша дорога из желтого кирпича изложена в двух вышеупомянутых документах.

<<Полжение о системе сертификации средств защиты информации>> - это просто инструкция, описывающая процесс сертификации и ответсвенность сторон.

Положение определяет состав участников системы сертификации средств защиты информации, создаваемой ФСТЭК России, а также организацию и порядок сертификации продукции, используемой в целях защиты сведений, составляющих государственную тайну или относимых к охраняемой в соответствии с законодательством Российской Федерации иной информации ограниченного доступа, являющейся государственным информационным ресурсом и (или) персональными данными, продукции, сведения о которой составляют государственную тайну, подлежащей сертификации в рамках указанной системы.

Список участников такой:
\begin{itemize}
    \item федеральный орган по сертификации;
    \item организации, аккредитованные ФСТЭК России в качестве органа по сертификации (органы по сертификации);
    \item организации, аккредитованные ФСТЭК России в качестве испытательной лаборатории (далее - испытательные лаборатории);
    \item изготовители средств защиты информации (вы находитесь вот тут)
\end{itemize}
Окей, что такое <<временное положение?>>. Ну, одно точно, нет ничего постояннее временного.

Положение устанавливает единый на территории Российской Федерации порядок исследований и разработок в области:
\begin{itemize}
	\item Защиты информации, обрабатываемой автоматизированными системами различного уровня и назначения, от несанкционированного доступа;
	\item Создания средств вычислительной техники общего и специального назначения, защищенных от утечки, искажения или уничтожения информации за счет НСД, в том числе программных и технических средств защиты информации от НСД;
	\item Создания программных и технических средств защиты информации от НСД в составе систем защиты секретной информации в создаваемых АС.
\end{itemize}
Положение определяет следующие основные вопросы:
\begin{itemize}
	\item Организационную структуру и порядок проведения работ по защите информации от НСД и взаимодействия при этом на государственном уровне;
	\item Систему государственных нормативных актов, стандартов, руководящих документов и требований по этой проблеме;
	\item Порядок разработки и приемки защищенных СВТ, в том числе программных и технических (в частности, криптографических) средств и систем защиты информации от НСД;
	\item Порядок приемки указанных средств и систем перед сдачей в эксплуатацию в составе АС, порядок их эксплуатации и контроля за работоспособностью этих средств и систем в процессе эксплуатации.
\end{itemize}

Вот так, как-то. Добро пожаловать в увлекательный мир нормативно-правовой документации в области защиты информации.









