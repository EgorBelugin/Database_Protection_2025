\section{Теоретические основы безопасности в СУБД}

2.1. Критерии защищенности БД
Критерии оценки надежных компьютерных систем (TCSEC). Понятие политики безопасности. Совместное применение различных политик безопасности в рамках единой модели. Интерпретация TCSEC для надежных СУБД (TDI). Оценка надежности СУБД как компоненты вычислительной системы. Монитор ссылок. Применение TCSEC к СУБД непосредственно. Элементы СУБД, к которым применяются TDI: метки, аудит, архитектура системы, спецификация, верификация, проектная документация. Критерии безопасности ГТК.

2.2. Модели безопасности в СУБД
Дискреционная (избирательная) и мандатная (полномочная) модели безопасности. Класси-фикация моделей. Аспекты исследования моделей безопасности. Особенности применения мо-делей безопасности в СУБД. Дискреционные модели: HRU, Take-Grant, Action-Entity, Wood. Мандатные модели: Bell-LaPadula, Biba, Dion, Sea View, Jajodia\&Sandhu, Smith\&Winslett, реше-точная. БД с многоуровневой секретностью (MLS). Многозначность.