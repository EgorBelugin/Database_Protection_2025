\section{Теоретические основы безопасности в СУБД}

\subsection{Критерии защищенности БД}

\subsubsection{Критерии оценки надежных компьютерных систем (TCSEC)}
<<Критерии безопасности компьютерных систем>> (Trusted Computer System Evaluation Criteria), получившие неформальное название <<Оранжевая книга>>, были разработаны Министерством обороны США в 1983 году с целью определения требований безопасности, предъявляемых к аппаратному, программному и специальному обеспечению компьютерных систем и выработки соответствующей методологии и технологии анализа степени поддержки политики безопасности в компьютерных системах военного назначения. В данном документе были впервые нормативно определены такие понятия, как <<политика безопасности>>, <<ядро безопасности>> (ТСВ) и т.д.

Предложенные в этом документе концепции защиты и набор функциональных требований послужили основой для формирования всех появившихся впоследствии стандартов безопасности.

В <<Оранжевой книге>> предложены три категории требований безопасности -- политика безопасности, аудит и корректность, в рамках которых сформулированы шесть базовых требований безопасности. Первые четыре требования направлены непосредственно на обеспечение безопасности информации, а два последних -- на качество самих средств защиты.

Рассмотрим эти требования подробнее:

\begin{enumerate}
	\item Политика безопасности
	\begin{itemize}
		\item \textbf{Политика безопасности}
		Система должна поддерживать точно определённую политику безопасности. Возможность осуществления субъектами доступа к объектам должна определяться на основе их идентификации и набора правил управления доступом. Там, где необходимо, должна использоваться политика нормативного управления доступом, позволяющая эффективно реализовать разграничение доступа к категорированной информации (информации, отмеченной грифом секретности — типа "секретно", "сов. секретно" и т.д.).
		\item \textbf{Метки} С объектами должны быть ассоциированы метки безопасности, используемые в качестве атрибутов контроля доступа. Для реализации нормативного управления доступом система должна обеспечивать возможность присваивать каждому объекту метку или набор атрибутов, определяющих  степень конфиденциальности (гриф секретности) объекта и/или режимы доступа к этому объекту.
	\end{itemize}
	\item Аудит
	\begin{itemize}
		\item \textbf{Идентификация и аутентификация} Все субъекты должен иметь уникальные идентификаторы. Контроль доступа должен осуществляться на основании результатов идентификации субъекта и объекта доступа, подтверждения подлинности их идентификаторов (аутентификации) и правил разграничения доступа. Данные, используемые для идентификации и аутентификации, должны быть защищены от несанкционированного доступа, модификации и уничтожения и должны быть ассоциированы со всеми активными компонентами компьютерной системы, функционирование которых критично с точки зрения безопасности.
		\item \textbf{Регистрация и учет} Для определения степени ответственности пользователей за действия в системе, все происходящие в ней события, имеющие значение с точки зрения безопасности, должны отслеживаться и регистрироваться в защищенном протоколе. Система регистрации должна осуществлять анализ общего потока событий и выделять из него только те события, которые оказывают влияние на безопасность для сокращения объема протокола и повышения эффективность его анализа. Протокол событий должен быть надежно защищен от несанкционированного доступа, модификации и уничтожения.
	\end{itemize}
	\item Корректность
	\begin{itemize}
		\item \textbf{Контроль корректности} Средства защиты должны содержать независимые аппаратные и/или программные компоненты, обеспечивающие работоспособность функций защиты. Это означает, что все средства защиты, обеспечивающие политику безопасности, управление атрибутами и метками безопасности, идентификацию и аутентификацию, регистрацию и учёт, должны находиться под контролем средств, проверяющих корректность их функционирования. Основной принцип контроля корректности состоит в том, что средства контроля должны быть полностью независимы от средств защиты.
		\item \textbf{Непрерывность защиты} Все средства защиты (в т.ч. и реализующие данное требование) должны быть защищены от несанкционированного вмешательства и/или отключения, причем эта защита должна быть постоянной и непрерывной в любом режиме функционирования системы защиты и компьютерной системы в целом. Данное требование распространяется на весь жизненный цикл компьютерной системы. Кроме того, его выполнение является одним из ключевых аспектов формального доказательства безопасности системы.
	\end{itemize}	
\end{enumerate}

\subsubsection{Понятие политики безопасности}
Политика безопаности -- это набор законов, правил, процедур и норм поведения, определяющих, как организация обрабатывает, защищает и распространяет информацию. Причём, политика безопасности относится к активным методам защиты, поскольку учитывает анализ возможных угроз и выбор адекватных мер противодействия.
\subsubsection{Совместное применение различных политик безопасности в рамках единой модели[None]}
\subsubsection{Интерпретация TCSEC для надежных СУБД (TDI)[None]}
\subsubsection{Оценка надежности СУБД как компоненты вычислительной системы[None]}
\subsubsection{Монитор ссылок}
Контроль за выполнением субъектами (пользователями) определённых операций над объектами, путем проверки допустимости обращения (данного пользователя) к программам и данным разрешенному набору действий.
Обязательные качества для монитора обращений:
\begin{itemize}
	\item Изолированность (неотслеживаемость работы)
	\item Полнота (невозможность обойти)
	\item Верифицируемость (возможность анализа и тестирования)
\end{itemize}
\subsubsection{Применение TCSEC к СУБД непосредственно[None]}
\subsubsection{Элементы СУБД, к которым применяются TDI: метки, аудит, архитектура системы, спецификация, верификация, проектная документация}
https://web.archive.org/web/20160303230445/http://ftp.fas.org/irp/nsa/rainbow/tg021.htm
\subsubsection{Критерии безопасности ГТК}
https://fstec.ru/component/attachments/download/293
\subsection{Модели безопасности в СУБД}
\paragraph{Дискреционная (избирательная) и мандатная (полномочная) модели безопасности}
\paragraph{Классификация моделей}
\paragraph{Аспекты исследования моделей безопасности}
\paragraph{Особенности применения моделей безопасности в СУБД}
\paragraph{Дискреционные модели: HRU, Take-Grant, Action-Entity, Wood}
\paragraph{Мандатные модели: Bell-LaPadula, Biba, Dion, Sea View, Jajodia\&Sandhu, Smith\&Winslett, решеточная}
\paragraph{БД с многоуровневой секретностью (MLS)}
\paragraph{Многозначность}