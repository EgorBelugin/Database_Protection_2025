\section{Механизмы обеспечения конфиденциальности в СУБД}

\subsection{Классификация угроз конфиденциальности СУБД}

\paragraph{Причины, виды, основные методы нарушения конфиденциальности}
\paragraph{Типы утечки конфиденциальной информации из СУБД, частичное разглашение}
\paragraph{Соотношение защищенности и доступности данных}
\paragraph{Получение несанкционированного доступа к конфиденциальной информации путем логических выводов}
\paragraph{Методы противодействия. Особенности применения криптографических методов}

\subsection{Средства идентификации и аутентификации}

\paragraph{Общие сведения}
\paragraph{Совместное применение средств идентификации и аутентификации, встроенных в СУБД и в ОС}

\subsection{Средства управления доступом}
\paragraph{Основные понятия: субъекты и объекты, группы пользователей, привилегии, роли и представления}
\paragraph{Виды привилегий: привилегии безопасности и доступа}
\paragraph{Использование ролей и привилегий пользователей}
\paragraph{Соотношение прав доступа, определяемых ОС и СУБД}
\paragraph{Метки безопасности}
\paragraph{Использование представлений для обеспечения конфиденциальности информации в СУБД}

\subsection{Обеспечение конфиденциальности путем тиражирования БД}
\paragraph{Формальная модель для обеспечения конфиденциальности БД с помощью тиражирования}
\paragraph{Архитектура и политика безопасности в модели SINTRA}

\subsection{Аудит и подотчетность}
\paragraph{Подотчетность действий пользователя и аудит связанных с безопасностью событий}
\paragraph{Регистрация действий пользователя}
\paragraph{Управление набором регистрируемых событий}
\paragraph{Анализ регистрационной информации}
