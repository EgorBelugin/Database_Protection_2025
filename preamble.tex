%%% Работа с русским языком
%\usepackage{cmap}					% поиск в PDF
%\usepackage{mathtext} 				% русские буквы в фомулах
\usepackage[T2A]{fontenc}			% кодировка
\usepackage[utf8]{inputenc}			% кодировка исходного текста
\usepackage[english,russian]{babel}	% локализация и переносы

\usepackage{extsizes} % Возможность сделать 14-й шрифт
\usepackage{indentfirst}
% \frenchspacing
\usepackage{setspace} % Интерлиньяж
\onehalfspacing % Интерлиньяж 1.5
\usepackage{geometry} % Простой способ задавать поля
	\geometry{top=30mm}
	\geometry{bottom=40mm}
	\geometry{left=30mm}
	\geometry{right=20mm}
%%% Дополнительная работа с математикой
\usepackage{amsmath,amsfonts,amssymb,amsthm,mathtools} % AMS
\usepackage{icomma} % "Умная" запятая: $0,2$ --- число, $0, 2$ --- перечисление

\usepackage{enumitem}

%%% Работа с таблицами
\usepackage{array,tabularx,tabulary,booktabs} % Дополнительная работа с таблицами
\usepackage{longtable}  % Длинные таблицы
\usepackage{multirow} % Слияние строк в таблице

%%%%%%%%%%
\usepackage{hyperref}
\usepackage{xcolor}
\definecolor{linkcolor}{HTML}{799B03} % цвет ссылок
\definecolor{urlcolor}{HTML}{799B03} % цвет гиперссылок
 
\hypersetup{%
    pdfstartview=FitH,%
    linkcolor=blue,%
    urlcolor=cyan,%
    colorlinks=true}

\usepackage{multicol} % Несколько колонок
\usepackage{graphicx}
\usepackage{babel,blindtext}
\usepackage{float}