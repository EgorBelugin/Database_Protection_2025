%%% Работа с русским языком
%\usepackage{cmap}					% поиск в PDF
%\usepackage{mathtext} 				% русские буквы в фомулах
\usepackage[T2A]{fontenc}			% кодировка
\usepackage[utf8]{inputenc}			% кодировка исходного текста
\usepackage[english,russian]{babel}	% локализация и переносы
\usepackage{listings}
\usepackage{extsizes} % Возможность сделать 14-й шрифт
\usepackage{indentfirst}
% \frenchspacing
\usepackage{setspace} % Интерлиньяж
\onehalfspacing % Интерлиньяж 1.5
\usepackage{geometry} % Простой способ задавать поля
	\geometry{top=30mm}
	\geometry{bottom=40mm}
	\geometry{left=30mm}
	\geometry{right=20mm}
%%% Дополнительная работа с математикой
\usepackage{amsmath,amsfonts,amssymb,amsthm,mathtools} % AMS
\usepackage{icomma} % "Умная" запятая: $0,2$ --- число, $0, 2$ --- перечисление

\usepackage{enumitem}

%%% Работа с таблицами
\usepackage{array,tabularx,tabulary,booktabs} % Дополнительная работа с таблицами
\usepackage{longtable}  % Длинные таблицы
\usepackage{multirow} % Слияние строк в таблице

%%%%%%%%%%
\usepackage{hyperref}
\usepackage[usenames,dvipsnames,svgnames,table,rgb]{xcolor}
\definecolor{linkcolor}{HTML}{799B03} % цвет ссылок
\definecolor{urlcolor}{HTML}{799B03} % цвет гиперссылок
 
\hypersetup{%
    pdfstartview=FitH,%
    linkcolor=blue,%
    citecolor=black,
    colorlinks=true}


%\hypersetup{				% Гиперссылки
%	unicode=true,           % русские буквы в раздела PDF
%	pdftitle={Заголовок},   % Заголовок
%	pdfauthor={Автор},      % Автор
%	pdfsubject={Тема},      % Тема
%	pdfcreator={Создатель}, % Создатель
%	pdfproducer={Производитель}, % Производитель
%	pdfkeywords={keyword1} {key2} {key3}, % Ключевые слова
%	colorlinks=true,       	% false: ссылки в рамках; true: цветные ссылки
%	linkcolor=red,          % внутренние ссылки
%	citecolor=black,        % на библиографию
%	filecolor=magenta,      % на файлы
%	urlcolor=cyan           % на URL
%}


\usepackage{multicol} % Несколько колонок
\usepackage{graphicx}
\usepackage{babel,blindtext}
\usepackage{float}

\usepackage{epigraph}
\setlength\epigraphwidth{.6\textwidth}

\usepackage{csquotes} % Инструменты для ссылок

\usepackage[backend=biber,bibencoding=utf8,sorting=none,maxcitenames=2,style=numeric,autocite=inline]{biblatex}

\usepackage[most]{tcolorbox} % для управления цветом
\definecolor{block-gray}{gray}{0.90} % уровень прозрачности (1 - максимум)
\newtcolorbox{grayquote}{colback=block-gray,grow to right by=-10mm,grow to left by=-10mm, boxrule=0pt,boxsep=0pt,breakable} % настройки области с изменённым фоном

\usepackage[normalem]{ulem} 

\usepackage{import} % Импорт из вложенных файлов
